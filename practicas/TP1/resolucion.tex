\documentclass{article}

\author{Franco Dalla Gasperina}

\title{Trabajo practico 1 - Programación distribuida y tiempo real}

\date{agosto 2025}

\begin{document}
	\maketitle
	
	\section{Ejercicio 1}
	1) Teniendo en cuenta la comunicación con sockets (puede usar tanto los
	ejemplos provistos como también otras fuentes de información, que se sugiere
	referenciar de manera explı́cita):
	\begin{itemize}
		\item a.- Identifique similitudes y diferencias entre los sockets en C y en Java.
		\item b.- ¿Por qué puede decirse que los ejemplos no son representativos del
		modelo c/s? Nota: corroborar con la clase donde se explica el modelo C/S.
		\item c.- ¿Qué cambio/s deberı́an hacerse para que “cliente” provisto funcione
		como “servidor” y el “servidor” provisto funcione como “cliente”? Nota:
		corroborar con la clase donde se explica el modelo C/S.
	\end{itemize}
	
	\section{Respuesta-1}
	\subsection{a}
		Las similitudes entre los sockets de C y Java son: 
		\begin{itemize}
			\item En ambos debe definirse si el socket utilizado sera para streams o datagramas.
		\end{itemize}
	
		Las diferencias entre ambos son:
		\begin{itemize}
			\item Java es un lenguaje de alto nivel. Por lo que hay mucha más abstracción en la realización de conexiones con sockets en este lenguaje en comparación a C. Por esta razón en C se puede, por ejemplo, no realizar algunas.
			\item en Java debe definirse el tipo de buffer (entrada o salida) a diferencia de C en el que no es necesario.
		\end{itemize}
	
	\subsection{b}
	Se discutió mucho al respecto sobre esto en las clases. La constante fue que no se cumplía el modelo c/s a causa de que el código que funciona como servidor es muy simple y no tiene en cuenta ciertas tareas que debería cumplir un servidor.
	Tales como:
	\begin{itemize}
		\item Concurrencia. El servidor de ejemplo no puede manejar varios pedidos al mismo tiempo.
		\item Escala. El servidor de ejemplo solo responde una vez y finaliza la ejecución.
	\end{itemize}
	
\end{document}